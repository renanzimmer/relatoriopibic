%do repo
% Definições Iniciais --------------------------------------------------
\documentclass[12pt]{report}
%\documentclass{abnt}
%\usepackage[a4paper]{geometry}
\usepackage[left=2.8cm, right=2.8cm, top=4cm]{geometry} % margens da folha

\usepackage[utf8]{inputenc} % caracteres com acento

\usepackage{epigraph}

\usepackage[english,portuguese]{babel}
\usepackage[T1]{fontenc} %tradução
\usepackage{lmodern}

\usepackage{url} % exibição de URL's
\usepackage{csquotes}
\usepackage{setspace} %espaçamento entre linhas

\usepackage{subfigure}
\usepackage{graphicx, multicol, latexsym, amsmath, amssymb} % permite inserção de imagens e fórmulas matemáticas

\graphicspath{{../pdf/}{../jpeg/}}
\DeclareGraphicsExtensions{.pdf,.jpeg,.png} % formatos aceitos

\usepackage{hyphenat} % hífens; veja a última linha antes do início do documento, corrigir erros de hifenização lá!

\usepackage{etoolbox}
\apptocmd{\sloppy}{\hbadness 10000\relax}{}{} % remove erros de underfull hbox em links

\usepackage{color} % permite a mudança de cor do texto

\usepackage{xcolor,soul,framed}
\colorlet{shadecolor}{yellow} % define cor do destacado como amarelo

\usepackage{silence}
\WarningFilter{caption}{Unsupported document class} % remove o erro no pacote 'caption'
\usepackage{caption} % permite a inserção de legendas em imagens

\usepackage{eqparbox} % facilita a criação de grupos

\usepackage{listings} % permite a escrita de códigos de programação

\usepackage{indentfirst} % identação depois de uma seção

\usepackage{tabularx} % tabelas com tamanho variável de coluna

\usepackage{array}

\makeatletter
\patchcmd{\chapter}{\if@openright\cleardoublepage\else\clearpage\fi}{}{}{}
\makeatother % Manter capítulos na mesma página

\usepackage{titlesec}% http://ctan.org/pkg/titlesec
\titleformat{\section}%
[hang]% <shape>
{\normalfont\bfseries\Large}% <format>
{}% <label>
{0pt}% <sep>
{}% <before code>
\titleformat{\subsection}%
[hang]% <shape>
{\normalfont\bfseries\large}% <format>
{}% <label>
{0pt}% <sep>
{}% <before code>
\renewcommand{\thesection}{}% Remove section references...
\renewcommand{\thesubsection}{}%... from subsections
\hyphenation{matemá-tica cientí-fico} % corrigir erros de hifenização aqui!

\usepackage{placeins}

\usepackage[
backend=biber,
style = abnt, % Sistema alfabético
% style = abnt-numeric, % Sistema numérico
% style = abnt-ibid, % Notas de referência
]{biblatex}
\addbibresource{bibliography.bib} % Arquivo de bibliografia

%formatacao da epigrafe
 \usepackage{etoolbox}
\makeatletter
\newlength\epitextskip
\pretocmd{\@epitext}{\em}{}{}
\apptocmd{\@epitext}{\em}{}{}
\patchcmd{\epigraph}{\@epitext{#1}\\}{\@epitext{#1}\\[\epitextskip]}{}{}
\makeatother

\setlength\epigraphrule{0pt}
\setlength\epitextskip{2ex}
\setlength\epigraphwidth{.8\textwidth}


% Início do documento --------------------------------------------------
\begin{document}
	
	% Cabeçalho
	\begin{center}
		\vspace*{-3cm}
		\textsc{Universidade Federal de Santa Catarina} \\
		\textsc{Centro de Filosofia e Ciências Humanas}
		
		\vspace{1cm}
		\rule{411pt}{1.3pt}
		\vspace{0.2cm}
		
		\Large \textbf{\textsc{Unidade da Ciência, Educação e Sociedade}}
		
		\rule{411pt}{1.3pt}
		\vspace{1cm}
		
		Aluno: Renan Hipólito Zimmer\\
		\vspace{0.5cm}
		Pesquisador Responsável: Ivan Ferreira da Cunha
		
		\vspace{1cm}
	\end{center}	
	
	% Resumo
	\vspace{-0.5cm}
	\section*{Resumo}
		Esta pesquisa teve o objetivo de relacionar o problema da unidade da ciência com a educação e sociedade em geral.
		Uma revisão bibliográfica em conjunto com discussões com o orientador e colegas em grupos de estudo salientou a contribuição das ideias de Feyerabend, John Dewey e Susan Haack para o tema.
		Ao final do projeto, tal conhecimento foi útil para a realização de um minicurso relacionando o momento atual causado pela pandemia e a filosofia da ciência, tal como para o início da elaboração de material de divulgação filosófica.
		
	\newpage
	\vspace*{-3cm}
	\onehalfspacing
	% Introdução
	\chapter*{Introdução}
	
		\vspace{-0.75cm}
		\section*{Contextualização}		
		Algumas discussões, como a sobre o papel do conhecimento científico na educação podem demandar, em uma análise mais cuidadosa, uma caracterização do que se entende por ciência.
		É evidente que as disciplinas científicas preocupam-se com diferentes objetos de estudo, e mesmo na prática, suas investigações são realizadas através de diferentes métodos.
		Contudo, se pressupusermos que o conhecimento científico como um todo deve possuir um destaque ou privilégio na educação, então também pressuporíamos que existe um certo tipo de unidade na ciência.
		Segundo \textcite{cunha-unidade} existem ao menos três respostas comuns para este problema: que a união dá-se pelo método científico; que existe um determinado corpo de conhecimento com características em comum; ou que é a origem histórica das atividades consideradas científicas o fator que as unifica.
		No entanto, existem outros aspectos de unidade na ciência na literatura, como a de que a ciência é o que se produz por uma maneira de pensar e a de que a ciência é uma maneira de conceber o mundo. A primeira está presente na obra de pragmatistas americanos e a segunda na de Neurath.
		Outros autores criticam a tentativa de caracterizar a ciência como una.
		\textcite{cartwright-dappled-world} enfatiza a desunidade na ciência ao criticar a universalidade das leis científicas.
		As críticas de Feyerabend à ideia de um método científico\footcite{feyerabend-against-method} também se direcionam à apresentação da ciência na educação de jovens em relação a outros tipos de conteúdos\footcite{feyerabend-science-free-society}.
		\textcite{haack-defending-science} defende a tese de que o adjetivo “científico” tornou-se um título honorífico e que as investigações científicas não são epistemologicamente privilegiadas em relação a outras consideradas não científicas. 
	
		\section*{Metodologia}
		
		Dois métodos foram utilizados para o desenvolvimento do trabalho:
		
		\begin{enumerate}
			\item Revisão bibliográfica orientada, em grupo de estudos e em reuniões.
			\item Participação em projetos de extensão na forma de grupos de discussão de textos filosóficos e elaboração de material didático para divulgação filosófica.
			Tais projetos foram organizados pelo orientador e contaram com a participação de outros estudantes de graduação. 
		\end{enumerate}

		\section*{Objetivos}
			Inicialmente, o objetivo deste projeto foi o de examinar as discussões sobre a unidade da ciência tendo como referencial a questão sobre o papel da ciência na sociedade.
			No decorrer do projeto, os objetivos específicos das leituras e discussões foram selecionados de forma a cumprir com o objetivo inicial, adequaram-se com os interesses do graduando e contribuir para a comunidade com materiais de divulgação acessíveis e relativos ao momento de isolamento ocasionado pela pandemia da COVID-19, tanto quanto à pesquisa.
			
			Em um primeiro momento, procurou-se atentar às críticas de Feyerabend para a ideia de que existe um determinado método científico que rege ou descreve as práticas científicas.
			Motivado pelos resultados obtidos com a leitura do texto \textit{Science in a Free Society} \cite{feyerabend-science-free-society}, o objetivo então passou a ser compreender algumas contribuições da filósofa Susan Haack, ainda em atividade.
			Influenciada por Feyerabend, Dewey e outros, ela apresenta uma abordagem pragmática e ponderada a respeito das práticas científicas e da relação de suas investigações com investigações cotidianas consideradas não científicas.
			
			Precursor de Susan Haack, o pragmatista americano John Dewey escreveu uma extensa obra acerca de aspectos da dimensão social da ciência, dentre eles, a relação entre investigação, valores, e seus impactos em políticas públicas e na política de forma mais ampla.
			O objetivo deste momento da pesquisa passou a ser compreender os fundamentos da obra Deweyana para servirem como instrumentos em um entendimento da relação entre ciência, educação e sociedade.
			
			A pandemia causada pelo novo coronavírus inicou-se durante este projeto de pesquisa e serviu como motivação para um estudo contextualizado sobre as relações entre a epidemiologia e a sociedade.
			As atividades, então, passaram a ter o objetivo de divulgação filosófica relacionanda à filosofia da ciência, mais especificamente da epidemiologia, em formato de vídeos e de um minicurso..
	
	% Discussão e Resultados
	\vspace*{-0.6cm}
	\chapter*{Discussão e Resultados}
	\vspace*{-0.75cm}
	
	As pesquisas durante o período deste projeto resultaram em um mini curso realizado na SEPEX (citar da maneira certa), em vídeos de divulgação filosófica e na compreensão de temas que servirão como fundamentos em pesquisas futuras do aluno.
	
	\section{Discussão das Leituras}
	
	\subsection{Unidade da Ciência}
	
		É possível sugerir a caracterização da ciência como una de diferentes formas. \textcite{cunha-unidade} aponta três discussões - citadas na introdução - sobre abordagens deste tema e sobre as quais a atenção das leituras durante esta pesquisa foi direcionada. No entanto, muitos, tal como Carnap, também discutem sobre a possibilidade das ciências serem unificadas através de uma teoria única.
		
	\subsection{A Crítica de Feyerabend à Proposta de Unidade Metodológica da Ciência}
		
		Na primeira metade do século XX, muitos autores, como \textcite{popper_conjeturas} propuseram um método científico específico que mais do que descrever, prescreve as práticas legitimamente científicas e as separa das não científicas.
		
		\subsubsection{Contra o Método}
			
			Feyerabend foi um filósofo austríaco também conhecido como ``o pior inimigo da ciência'', designação atribuída devido às suas críticas à imposição do conhecimento científico acima de qualquer outro. Influenciado por Mill, defendia que silenciar a oposição em um debate, independentemente de quão absurdas suas teses parecessem, levaria a uma sustentação de suas próprias teses dogmaticamente.
			Além do mais, ao debatermos com adversários, fortalecemos nossos próprios pontos de vista. Os argumentos de Feyerabend não são meramente contra censura. Sua ideia é que diferentes pontos de vista pudessem ser proliferados e competir na sociedade, assim sendo, devería-se encorajar diferentes pontos de vista a florescer.
			
			Um dos pontos que o austríaco mais sustentou é o de que não existe um método científico universal\footnote{\cite{feyerabend-against-method}}. Para tal, pontuou momentos da história da ciência em que cientistas não utilizaram nenhum método específico e, por consequência disso, houve progresso científico, fato que não aconteceria caso seguisse algum método científico. Ao contrariar a tese de que há uma unidade metodológica na ciência, Feyerabend afirma que o único método que funciona em todas as situações na ciência é ``anything goes'' (tudo vale). Apesar de cientistas seguirem métodos, eles serão diferentes em diferentes disciplinas e comunidades científicas. Além disso, sempre há a possibilidade de uma inovação acarretar no abandono de métodos então empregados.
			
			Feyerabend discutiu a defesa do Heliocentrismo de Galileu para sustentar suas teses em ``Against Method''.
			Argumentou que se existe um método científico universal, então ele seria seguido pelos grandes cientistas. Galileu era um grande cientista, todavia os métodos que usara violavam os que até então eram seguidos. Logo, não há um método científico universal. O heliocentrismo era inconsistente com a metafísica da época profundamente derivada da aristotélica, aceitar a nova teoria resultaria novas teorias em diversas áreas do conhecimento. Além disso, naquela época, o heliocentrismo foi refutado pela observação dos astros sem o uso do telescópio. O uso sugerido do telescópio por Galileu sofreu críticas. Não haviam bons motivos para se confiar no telescópio ao considerar as teorias vigentes.
			
			Apesar das teorias de Galileu, de diferentes áreas do conhecimento, serem refutadas individualmente, elas eram coerentes entre si. Esse momento na história da ciência foi utilizado por Feyerabend para ilustrar a tese de que o progresso científico depende da proliferação de diferentes teorias competindo. Por consequência, o austríaco argumenta que a ciência não é una. Existem várias atividades que chamamos de ciência que não tem interesses em comum, então também não há uma distinção clara entre ciência e não ciência. Os debates éticos em que certos valores na prática científica são defendidos ou atacados também colocam em voga uma competição de aspectos não apenas atribuídos às teorias científicas, mas que são importantes para o desenvolvimento da ciência.
			
		\subsubsection{Democracia e Ciência}
		
			Segundo Feyerabend, ao emergir durante o iluminismo, a ciência desafiou o controle e a tirania do cristianismo. No entanto, nos tempos recentes, ela se tornou tão tirânica quanto a Igreja fora. Para o filósofo, tal afirmação evidencia-se na influência dos cientistas em políticas públicas, nos grandes fundos monetários recebidos para projetos técnicos e na importância aos fatos científicos no sistema educacional.
		
			A ciência tornou-se uma ameaça para a democracia à medida em que as outras tradições são silenciadas por consequência de seu poder. Feyerabend considera que a ciência é poderosa devido ao seu sucesso. No entanto, o pensamento de que a ciência é superior às outras tradições e não poder ser desenvolvida pela interação com estas frequentemente resulta na ideia de que a ciência deveria ser aceita e servida de base para a sociedade.
		
			No entanto, o que torna a ciência tão especial ao ponto de conceder sua superioridade? Feyerabend rejeita a ideia de que um método específico tornaria a ciência superior. Outra resposta seria a de que a ciência alcançou resultados muito mais impressionantes do que outras tradições, ou seja, que forneceu um enorme corpo de conhecimento e habilidades tecnológicas ao mundo. No entanto, o que é julgado como bom resultado varia de tradição para tradição. Além disso, outras tradições são eliminadas quando conflitam com a ciência. Antes do iluminismo, a humanidade desenvolveu tecnologias sofisticadas de construção e se adaptou por quase todos os biomas terrestres. Técnicas desenvolvidas pelo que consideramos hoje como medicinas alternativas servem como tratamento para inúmeras enfermidades e, mais do que isso, servem como inspiração e são fontes de estudo das disciplinas científicas. Assim sendo, os resultados da ciência não são consequências dos resultados da ciência por si só. Além do mais, os resultados da ciência não são fixos, especialistas divergem sobre teorias e a modificam frequentemente além de exagerarem sobre seu próprio conhecimento e criticarem campos que eles não compreendem bem. Devido a estes problemas, Feyerabend conclui que a ciência não é superior às outras tradições e que pode ser desenvolvida e melhorada através da interação com elas.
			
			No entanto, mesmo que a ciência fosse superior e que as outras tradições não valessem a pena, o austríaco afirmava que seria fundamentalmente antidemocrático tornar a ciência como a base para a sociedade. Em uma sociedade democrática, as pessoas deveriam ter o direito de seguir a tradição que quisessem, já que cada indivíduo poderia julgar da melhor forma sua forma de viver. Sendo assim, a população deveria decidir qual seria o papel de diferentes tradições na sua forma de viver. Assim como a religião é separada do Estado, a ciência também deveria ser. Uma sociedade livre e democrática deveria permitir criacionistas e míticos, por exemplo, a influenciar nas políticas públicas.
		
	\subsection{Susan Haack}
	
		\epigraph{``(...)this hononfic usage
			stands in the way of a straightforward acknowledgement
			that science — science, that is, in the descriptive sense — is
			neither sacred nor a confidence trick''}{Susan Haack, Defending Science whithin reason: Between Scientism and Cynism}
	
		A caracterização de ciência proposta por Susan Haack é realizada posicionando este empreendimento humano entre duas visões extremas, a daqueles que a autora chama de ``novos cínicos'', pessoas que desprezam a atividade científica e seus produtos, e os que denomina ``velhos deferencialistas'', os que tendem a aceitar sem questionar as afirmações científicas. 
		
		Em ``Defending science-within reason: Between scientism and cynicism''\footcite{haack-defending-science}, Haack pontua que o termo ``científico'' é utilizado frequentemente para atribuir um valor positivo. Apesar dos sucessos das ciências naturais servirem de um bom motivo para este uso, ao qualificarmos o que é ``científico'' como sendo bom, obscurecemos algo que não é claro para a maioria da população, que existem investigações não consideradas científicas que também possuem méritos. Assim sendo, a filósofa qualifica as ciências ao comparar e aproximar suas investigações com as de outras disciplinas.
	
		Susan Haack afirma, Seguindo essa forma de enxergar a ciência, que ela não é sagrada a ponto de ser infalível ou perfeita, nem um golpe de confiança, já que as ciências naturais possuem mérito ao serem consideradas muito bem sucedidas enquanto um empreendimento humano cognitivo. Partindo daí, a britânica investiga questões que objetivam caracterizar mais detalhadamente a atividade científica, e duas delas interessam a quem lê sua obra a fim de relacioná-la com o problema da unidade nas ciências, a saber, ``como as ciências se diferenciam e o que elas tem em comum?'' e ``como investigações não científicas se aproximam das científicas?''.
	
		Ao contrário do que a tradição da filosofia da ciência afirmou durante o século passado, Susan Haack não acredita que as investigações científicas são epistemologicamente privilegiadas em relação a outros tipos de investigação. A ciência não é nada mais do que o produto de uma especialização das investigações cotidianas, de senso comum, servida de uma ajuda típica das comunidades científicas e de ferramentas - como métodos de revisão por pares - que, em conjunto, tornam seus resultados confiáveis.
		
		Por estes motivos, as ações que formam as investigações científicas não são tão superiores quanto investigações do senso comum sem as mesmas ajudas e ferramentas, tal como defendido por alguns defensores da existência de um ideal de racionalidade intrínseco à ciência mas não a outras atividades.

		\textcite{haack-defending-science} afirma que por mais que não haja uma unidade das ciências naturais e sociais, elas estão integradas. As ciências sociais são formadas por investigações com ajudas e ferramentas semelhantes às naturais, no entanto, seu objeto de estudo é diferente. Apesar do mundo social fazer parte do mundo natural, a intencionalidade típica de desejos, comportamentos e crenças, por exemplo, não é explicada de forma satisfatória apenas com um vocabulário físico. O uso da estatística é, nos dois campos, aplicado. No entanto, de forma tal que uma transposição dos métodos de investigação da física em que a estatística está presente para a antropologia não seria de grande valia. A estatística aplicada na antropologia apresenta-se com métodos e tratamentos de dados distintamente. Outro ponto que a autora salienta é que os resultados e aplicações que exemplificam o sucesso das ciências naturais em relação ao das sociais não devido ao fato de uma disjunção das duas famílias de ciências. Haack sugere que o número e a complexidade das variáveis é responsável pela dificuldade da manipulação e predição do objeto de estudo das ciências sociais. A metereologia é o caso de uma ciência natural que, tal como ciências humanas, tem dificuldade em coletar e tratar todos os dados relevantes para gerar previsões extremamente precisas como as de ciências naturais cujos modelos comportam-se adequadamente com dados mais simples.
	
	\subsection{John Dewey}
	
		A obra do pragmatista americano John Dewey apresenta diversas maneiras de abordar as relações entre ciência e educação. Assim como Susan Haack concordou posteriormente, o autor negara que a prática científica possuía alguma qualidade epistêmica intrínseca privilegiada em relação às investigações do senso comum. Dewey entendia a prática científica como a atividade de investigação empírica mais inteligente e especializada possível de acordo com as evidências disponíveis no momento, e esta não estava separada das demais investigações cotidianas, mas se conectavam continuamente. Assim sendo, ensinar ciência é ensinar a investigar.
		
		O conceito de investigação proposto pelo norte americano deve ser compreendido ao tomarmos como pressuposto que toda a experiência ocorre em uma interação entre organismo e meio no qual um problema prático pretende ser resolvido. Dewey propôs um esquema simplificado e didático para compreendermos as fases da investigação, que são assim descritas:
	
			\begin{enumerate}
				\singlespacing
				\item A fase inicial começa com um sentimento de algo errado, uma dúvida, uma situação indeterminada também chamada, por vezes, de situação problema.
				\item A formulação de um problema que não existia antes do processo de investigação.
				\item A construção de hipóteses, de maneiras de testá-las e das possíveis consequências.
				\item O raciocínio envolvendo as hipóteses testadas que avalie suas implicações, caso seja constatada alguma contradição, há um retorno à fase 3.
				\item A corroboração da hipótese por meio da observação a partir da experimentação. Processo no qual, segundo dewey, uma situação indeterminada é convertida em uma “situação determinada”.
		\end{enumerate}
	
		\onehalfspacing 
		Durante e após investigações, frequentemente outras situações indeterminadas originam-se resultando em novas investigações. E como para compreender o conhecimento obtido como produto destas investigações é, para Dewey, necessário que se compreenda o processo que o resultou, evidencia-se a importância da análise histórica de como aquele conhecimento ocorreu. Assim sendo, o conhecimento científico é aquele que é estabelecido como fonte para investigações futuras, não podendo ser fixado de modo a não estar sujeito a revisões por estas.
		
		Tal como o conhecimento científico, através de investigações é possível avaliar os resultados de nossas práticas levando em consideração que estas eram meios para alcançarmos certos fins, isto é, objetivos que valorizamos ou não. Isso ocorre porque, para o americano, valores também ocorrem através da experiência, na relação entre organismo e meio. Dessa forma, ao avaliarmos criticamente se os resultados de nossas ações alcançaram os fins esperados, reavaliamos as atitudes tomadas e os próprios valores anteriores. Esta reavaliação crítica realizada de forma honesta, inteligente e democrática permite com que a sociedade desenvolva tecnologias sociais com a intenção de solucionar os problemas mais latentes na população. Assim sendo, a criação de políticas públicas, por exemplo, levaria em conta as melhores evidências disponíveis - como teorias das ciências sociais - de forma que uma investigação sobre seus resultados pudesse retroalimentar a concepção da sociedade sobre a forma como os cidadãos se relacionam para possivelmente aprimorar estas relações em novas políticas públicas definidas por futuras investigações.
		
		Portanto, para Dewey, uma educação que preparasse seus cidadãos para a vida democrática deveria ensinar os jovens a investigar e buscar em experiências passadas evidências para as decisões futuras. Assim sendo, a sociedade aprenderia a valorizar as boas investigações e compreenderia a ciência e sua importância para a educação de forma mais adequada em relação àquela em que o currículo escolar é a apresentação de um aglomerado de conteúdos obtidos não através da investigação, mas da memorização.
	
	\section{Filosofia da Epidemiologia}
	
	 	Ao início da quarentena provocada pela pandemia do vírus Sars-Cov-2, a leitura do livro ``Philosophy of Epidemiology'' \cite{broadbent} resultou em discussões em grupo de estudo, em vídeos de divulgação filosófica, e em um curso ministrado durante a SEPEX.
	 	
	 	Broadbendt defende que a Epidemiologia é uma ciência com características incomuns entre as ciências que são frequentemente objetos de análise pela tradição da filosofia da ciência. Por exemplo, a epidemiologia não possui uma teoria de fundo, ao contrário de outras ciências cujos experimentos não raramente são apenas observacionais. A Epidemiologia possui a característica de desenvolver seus métodos para a finalidade de aprimorar o estado de saúde em populações humanas. Assim sendo, as análises a respeito da distribuição de fatores determinantes de condições de saúde, são abastecidas com informações relacionadas a diversas disciplinas científicas que possam ajudar de alguma forma, desde conhecimento sobre genética advindos da biologia ou de demografia da geografia.
	 	
	 	A produção de materiais de divulgação realizada teve como objetivo popularizar contribuições da filosofia da ciência para a compreensão da atividade da epidemiologia\footnote{Os vídeos estão disponíveis em \textcite{youtube}}.
	 	
	 	Durante a XVIII Semana da pesquisa, ensino e extensão da UFSC, como resultado da pesquisa, foi ministrado o minicurso intitulado ``Introdução filosófica à epidemiologia: pensar faz bem à saúde''.
	 	O curso, através de uma plataforma de conferências virtuais, apresentou tanto materiais técnicos situando o problema da indução na prática da epidemiologia, quanto uma caracterização de diferentes posicionamentos do senso comum para com a ciência utilizando conceitos propostos por Susan Haack.
		
	% Considerações finais
	\vspace*{-0.6cm}
	\chapter*{Considerações finais}
	\vspace*{-0.75cm}
	
	As leituras e discussões realizadas durante o período da pesquisa, apesar das diversas motivações e objetivos, culminaram em um corpo de conteúdo cujas partes se inter-relacionam ao levarmos em consideração o tema ``unidade da ciência, educação e sociedade''.
	
	Tanto Feyerabend quanto Dewey, tal como explicitado anteriormente, escreveram sobre o papel do conhecimento científico na educação.
	Enquanto o austríaco criticava o papel tirano da ciência na educação, o americano sugeria que uma postura científica deveria ser desenvolvida nas escolas.
	No entanto, conforme \textcite{cunha_sci_and_edu}, é possível que as críticas de Feyerabend e as ideias de Dewey sejam compatíveis quando compreendemos o contexto histórico de cada autor e ao diferenciar os aspectos da educação relativos à formação profissional dos jovens de um caráter mais geral da educação destinado a preparar as pessoas a compreender o valor de diferentes propostas e tradições.
	
	Um paralelo entre a tecnologia social como apresentada por John Dewey e a prática da Epidemiologia é facilmente identificado.
	Isso ocorre devido ao objetivo desta ciência tal como descrito por Broadbendt: aprimorar a saúde de populações humanas.
	A escolhas dos métodos utilizadas para esta finalidade e as diferentes disciplinas que colaboram para tal adequam-se à descrição de Dewey da prática científica, já que ocorre constantemente uma revisão de meios para atingir determinados fins almejados.
	
	Como supracitado, os conceitos cunhados por Susan Haack foram utilizados durante o minicurso para tratar de posicionamentos do senso comum ante à ciência.
	O momento de pandemia evidenciou a postura de muitas pessoas caracterizada por uma grande desconfiança das afirmações realizadas por comunidades científicas.
	Muitas ideias compartilhadas, como a sobre o vírus ter sido espalhado de propósito para que a China tirasse vantagem da situação, ou sobre \textit{lockdowns} serem jogadas políticas que, de fato, não ajudam a conter o contágio, fizeram parte do repertório de opiniões enunciadas por diversas pessoas.
	O novo cinismo, segundo a definição da filósofa da ciência carrega uma postura crítica sobre a comunidade científica de forma a frequentemente ser acusada pelos cínicos de serem desonestos ao afirmarem respeito por evidências em troca de poder.
	Assim sendo, podemos aproximar aquelas opiniões demasiadamente desconfiadas sobre o consenso científico da caracterização realizada por Haack dos novos cínicos.
	No entanto, posicionamentos cientificistas também estiveram presente.
	Um dos sinais de cientificismo, segundo \textcite{haack-six-signs}, é o uso honorífico de termos relacionados a ``científico''.
	Durante as trocas de opinião, por vezes percebeu-se o uso de opiniões isoladas de cientistas para defender opiniões heterodoxas.
	Apesar de Broadbendt salientar que a opinião de especialistas é uma evidência a ser considerada utilizando o movimento ``medicina guiada por evidências'' como referência, estas constituem a classe das evidências mais fracas em relação a estudos epidemiológicos.
	Deste modo, uma postura cientificista poderia ser adotada mesmo por pessoas que criticam consensos científicos.
	
	O conhecimento obtido durante a pesquisa motivou uma nova pesquisa que, durante a graduação, culminará em um trabalho de conclusão de curso a respeito da relação, mais especificamente, entre o conhecimento científico e a educação.
	
	\newpage
	\vspace*{-3cm}
	% Bibliografia
	\printbibliography
	
\end{document}